\tongjisetup{
  %=========
  % 中文信息
  %=========
  ctitle={基于一种简支梁平面问题解析解和\\有限元解的比较分析},
  cheadingtitle={基于一种简支梁平面问题的解析解和有限元解比较分析},   
  cauthor={唐霖辉},  
  studentnumber={2410486},
  ccategories={工学},
  cmajorfirst={土木工程},
  cmajorsecond={结构工程},
  cdepartment={土木工程学院},
  csupervisor={何敏娟 教授}, 
  cresearchfield={大跨木结构},
  cassosupervisor={任晓丹 教授},
  %=========
  % 英文信息
  %=========
  etitle={Based on the comparative analysis of the analytical solution and the finite element solution of the plane problem of simply supported beams}, 
  eauthor={Linhui Tang},
  ecategories={Gong Xue},
  emajorfirst={Civil Engineering},
  emajorsecond={Structural Engineering},
  edepartment={School of Civil Engineering},    
  esupervisor={Prof. Minjuan He},
  eassosupervisor={Prof. Xiaodan Ren},
  eresearchfield={Large-span timber construction},
  }
  %=========
  % 中英文摘要和关键字
  %=========
\begin{cabstract}  
  弹性力学的平面问题及其解法可以作为弹性力学到弹塑性力学的一个过渡。在众多解法中,半逆解法一方面考虑了人的直观想象能力,另一方面充分利用了双调和方程。本研究对简支梁在受均布荷载作用下的平面问题进行了讨论。首先利用半逆解法,求解出了该问题的解析解,得到了简支梁应力、应变和位移的解析方程。接着在商业有限元软件~ABAQUS~中建立该问题的梁单元模型,得到了简支梁应力、应变和位移的数据及云图。最后比对解析解和有限元解的结果,发现有限元建立的模型与实际问题具有一致性,但在离散方法、边界条件上对实际问题进行了简化。而解析解完全将实际问题考虑为了平面应力问题,忽略了平面外真实存在的应力。因此在分析弹塑性力学问题时,既要考虑到解析解的简化方法,又要考虑到有限元建模时的各类误差。最后对于实际工程可能涉及到的弹性平面问题,本研究评估了两种方法的适用性与局限性,同时为相关领域的非平面问题或弹塑性问题研究提供参考。
\end{cabstract}

\ckeywords{弹性力学, 平面问题, 半逆解法, 有限元分析}

\begin{eabstract}
  The plane problem of elastic mechanics and its solution can be used as a transition from elastic mechanics to elastoplastic mechanics. Among the many solutions, the semi-inverse solution takes into account the intuitive imagination of human beings on the one hand, and makes full use of the two-tone sum equation on the other hand. In this study, the planar problem of simply supported beams under uniform load is discussed. Firstly, the analytical solution of the problem is solved by using the semi-inverse solution method, and the analytical equations of stress, strain and displacement of simply supported beams are obtained. Then, the beam element model of the problem is established in the commercial finite element software ABAQUS, and the data and contours of the stress, strain and displacement of the simply supported beam are obtained. Finally, comparing the results of the analytical solution and the finite element solution, it is found that the model established by the finite element is consistent with the practical problem, but the practical problem is simplified in the discrete method and boundary conditions. The analytic solution completely considers the actual problem as a plane stress problem, ignoring the real stress outside the plane. Therefore, when analyzing elastoplastic mechanics problems, it is necessary to consider not only the simplification method of the analytical solution, but also the various errors in finite element modeling. Finally, for the elastic plane problems that may be involved in practical engineering, this study evaluates the applicability and limitations of the two methods, and provides a reference for the research of non-planar problems or elastoplastic problems in related fields.
\end{eabstract}

\ekeywords{elastic mechanics, plane problems, semi-inverse solutions, finite element analysis}