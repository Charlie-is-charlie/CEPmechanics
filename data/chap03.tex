\chapter{简支梁平面问题的有限元解}
\label{cha:FEAsolution}
本章节使用商业有限元软件~ABAQUS~求解简支梁平面问题,即通过数值方法得到有限元解。对模型的信息和参数化建进行简述,并对计算得到的简支梁的应力、应变、位移场量进行分析和可视化。
\section{有限元模型的建立}
\subsection{单元类型选择}
在~ABAQUS~中,图\ref{fig:load}所示的简支梁可以使用梁单元(Beam Element)或实体单元(Solid Element)来模拟,如图\ref{fig:models}所示。
\begin{figure}[htbp]
    \centering
	\includegraphics[width=1\textwidth]{figure3}
    \caption{ABAQUS中梁单元和实体单元模型}
    \label{fig:models}
\end{figure}
其中梁单元适用于分析中如果梁的横截面几何尺寸远小于其长度时的情况,能较好地模拟结构的整体变形,适和得出简支梁的位移分布。因为梁单元直接假设了梁的弯曲和拉伸行为,通过简化的刚度矩阵,能够快速、准确地反映结构的整体位移。它省略了横截面内部的应力细节,只提供关键的全局位移,降低了误差。
实体单元通过大量的网格划分来精确模拟简支梁,适用于任意形状的简支梁,能较好地模拟简支梁网格点处具体的应力、应变和位移分量并绘制云图。实体单元模型也能反应荷载施加点或约束处的应力集中问题。

因此本研究首先建立简支梁的实体单元模型,得出应力、应变、位移的有限元解,然后建立对应的梁单元模型,计算该模型下的位移有限元解,并讨论两种模型的计算误差。
\subsection{模型参数设置}
模型的形状参数与材料参数的设置与\ref{cha:visualization}章一致。但是三维有限元模型需要考虑梁的厚度(截面宽度)。在本章节先假定梁的厚度为~100mm,下一章节对梁厚度的影响进行参数化分析。在网格划分模块中,实体单元模型选择~C3D8R~单元,最小网格尺寸设置为~50;梁单元模型选择~B31~单元,最小网格尺寸设置为~20,其余参数不变。最后创建静力分析步并设置应力、应变和位移场输出。
\subsection{边界条件设置}
简支梁的支座仅约束梁的垂直位移,梁端可自由转动。为使整个梁不产生水平移动,在一端加设水平约束,该处的支座称为铰支座,另一端不加水平约束的支座称为滚动支座。因此,在梁单元模型中,通过位移/转角约束,限制左端节点的~U1、U2、U3~分量,限制右端节点的~U1、U3~分量,通过线荷载的形式施加均布荷载~4~N/mm;在实体单元模型中,对梁的中性面两侧单元进行约束,同样限制左端节点的~U1、U2、U3~分量,限制右端节点的~U1、U3~分量,通过应力的形式施加均布荷载~4~MPa。将上述边界条件设置在静力分析步中,提交工况并得出结果。
\section{参数化建模}

\section{数值解的分析与讨论}
\subsection{参数分析}
\subsection{结果讨论}