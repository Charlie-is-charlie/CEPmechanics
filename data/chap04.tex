\chapter{两种解法的对比分析}
\label{cha:comparison}
对比\ref{cha:discussion1}章节和\ref{cha:discussion2}章节的结论,可以看出解析解和有限元解在结果精度上存在差异。解析解采用半逆解法推导出应力和位移分布的理论表达式,其优势在于公式简明、结果直接,在近似为平面应力的情况中更为精确。但这种方法忽略了截面高度对应力分布的影响,从图\ref{fig:fea}中可以看出,横向应力在梁的边缘区域出现了明显的应力集中现象,这与实际情况更为符合,而图\ref{fig:EPMplot}中横向应力并未出现该现象,因此解析解在支座附近对真实情况存在一定误差。

在弹性力学解析解中,支座处被简化为理想的铰支座和滑动支座,不会发生应力集中,因为这种解法假设梁和支座之间不存在摩擦或局部变形。而在有限元模拟中,支座通常通过网格和边界条件进行离散化,由于离散的节点约束无法完美模拟解析解中的理想支座,会在支座位置产生应力集中效应。

在有限元解中,梁单元模型的结果的最大挠度为~6.086~mm,最大弯曲应力为~48.000~MPa。在解析解中,模型的最大挠度为~6.058~mm,最大弯曲应力为~48.798~MPa,两者十分接近。因此在远离支座端的情况下,利用半逆解法得到的最大挠度和最大弯曲应力具有可靠性。

其次,从图\ref{fig:BEAMfea}中可以看出,梁单元的支座处弯矩不为~0,这与简支梁的支座弯矩为~0~矛盾。ABAQUS~中使用的梁单元(如~B23~或~B31~等)是基于弯曲的假设来计算的,这意味着在梁单元的局部区域,弯矩可能有非零值。在简支梁的支座位置,梁单元的刚度和支座的约束结合可能产生弯矩。在有限元分析中,由于数值求解方法的逼近特性,即使在理论上支座的弯矩应为零,实际的计算可能会因为网格划分、近似方法等导致支座弯矩存在微小的非零值。因此如果要考虑到支座处的受力情况,应采用实体单元模型。