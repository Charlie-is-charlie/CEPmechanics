\chapter{引言}
\label{cha:intro}
\section{研究背景及意义}
弹性力学与弹塑性力学是结构工程中不可或缺的基础学科。简支梁的平面应力问题是弹性力学中的典型问题之一,其分析方法为工程中复杂结构的力学行为提供了可靠的理论依据。随着工程材料及结构形式的多样化,尤其是在高层建筑和桥梁等大跨度结构设计中,对简支梁应力和位移精确分析的需求愈发重要。本研究以简支梁受均布荷载的平面问题为研究对象,比较解析解与有限元解的差异,分析二者在实际工程中的适用性,以便为结构设计中的方法选择提供参考依据。
\section{国内外研究现状}
针对简支梁的平面应力问题,国内外学者提出了多种求解方法,包括手算方法,数值方法和深度学习方法。传统的解析方法因其推导严谨且便于理论验证而得到广泛应用。任晓丹和冯德成\cite{gctsxyl}在对弹塑性力学中的逆解法和半逆解法进行了详细讨论,并应用于典型的工程力学问题,为本文解析解的推导提供了理论支持。李华东等\cite{LXYS201201011}讨论了简支梁在均布荷载作用下的弯曲应力分布,采用半逆解法求解了相关平面应力问题。但该方法在面对复杂边界条件时具有一定局限性。

随着计算机技术的发展,有限元方法(Finite Element Method, FEM)为复杂结构分析提供了有效的数值解法,能够适应多种复杂边界条件,并在数值精度上获得显著提升。近年来,有限元方法被广泛应用于简支梁的受力分析,尤其是在复杂荷载和边界条件的情况下,已成为模拟简支梁力学性能的主流数值方法。Chakraborty~和~Biswas\cite{chakraborty2018finite}研究了有限元法在简支梁受均布荷载下的应力和挠度分布,指出网格划分对计算结果的精度有显著影响。Yang~和~Feng\cite{yang2020static}提出了简化的有限元模型,用于简支梁的静力学和动力学分析,优化了网格划分和单元类型选择,以提高计算效率。Zienkiewicz~等\cite{zienkiewicz2013finite}的经典著作详细阐述了有限元法的基本理论和应用,涵盖了简支梁等常见结构的建模方法。同时,Jiang~和~Huang\cite{jiang2019effect}通过数值模拟分析了边界条件对简支梁有限元模型的影响,指出合理的边界条件设置是保证模型精度的关键。这些研究为简支梁的有限元模拟提供了重要的理论和实践指导。

人工智能(Artificial Intelligence, AI)和深度学习(Deep Learning)的引入为弹塑性力学问题提供了新的解决途径。Raissi~等\cite{raissi2019physics}提出了物理信息神经网络(Physics-Informed Neural Networks, PINNs),用于求解偏微分方程的正问题和反问题。Karniadakis~等\cite{karniadakis2021physics}讨论了~PINNs~在力学分析中的应用潜力,为本研究展望章节关于~AI~在力学中的应用提供了支撑。该方法被用于模拟复杂力学场下的应力和应变分布,在不依赖大规模网格划分的情况下能够获得高精度结果,成为了固体力学研究的热点。
\section{研究内容}
本文首先对简支梁在均布荷载作用下的平面应力问题进行简要介绍,并运用半逆解法推导出解析解,得出梁的应力、应变、位移场的表达式。然后,利用~ABAQUS~建立有限元模型,通过实体单元模拟梁的受力情况,得到有限元解。在此基础上,比较两种解法的结果,分析解析解和有限元解在离散方法和边界条件上的不同,并探讨其在工程应用中的适用性。
