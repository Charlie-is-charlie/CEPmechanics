\chapter{结论与展望}
\label{cha:conclusion}
\section{结论}
本研究通过对比解析解和有限元解对简支梁平面应力问题的求解,得出以下主要结论:

(1)结果精度:在支座端,实体单元的有限元解能更好的模拟实际情况,反映出了支座处的应力集中现象,而解析解和梁单元模型的有限元解在支座端的受力情况具有较大的误差。在远离支座端(跨中部位),解析解和梁单元模型的有限元解具有良好的精度,而实体单元的有限元解相较于前两者具有约~5\%~至~7\%~的误差。

(2)适用范围:解析解适合简单结构和荷载的情况,如本文研究的简支梁模型。其中半逆解法需要假定应力函数~$\phi$~的形式, 适用于~$\phi$~可以合理假设并实现后续求解的结构。有限元解在此基础上能适用于复杂结构和多种边界条件,是工程实际中广泛应用的方法。在~ABAQUS~中,梁单元模型适合分析结构中梁的整体变形和弯曲正应剪应力;梁单元模型适合分析结构中梁的各向应力应变状态。

(3)工程应用:在实际工程中,两种方法可以互补。解析解用于理论验证和快速评估,有限元解用于深入分析复杂结构,为结构设计提供更为可靠的依据。如果复杂结构的解析解很难求解,则需要借助数值方法对实际问题进行求解,通过有限元软件合理建模、分析后得到的结果会对实际工程更有帮助。

\section{展望}
随着计算技术的发展,深度学习、机器学习等技术在固体力学领域的应用潜力巨大。PINNs~等智能算法已在一些复杂流体、固体力学问题中显示出良好的应用前景。未来的研究可以结合解析解和有限元解,通过深度学习实现简支梁、悬臂梁等简单结构的受力分析,进一步研究在桁架、刚架等复杂结构分析中的应用,并将有助于推动结构工程的智能化发展。